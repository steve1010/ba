\documentclass[fontsize=12pt, paper=a4, headinclude, twoside=false, parskip=half+, pagesize=auto, numbers=noenddot, plainheadsepline, open=right, toc=listof, toc=bibliography]{scrreprt}
% PDF-Kompression
\pdfminorversion=5
\pdfobjcompresslevel=1
% Allgemeines
\usepackage[automark,footsepline,plainfootsepline,headsepline]{scrpage2} % Kopf- und Fußzeilen
\usepackage{amsmath,marvosym} % Mathesachen
\usepackage{mathtools} % Mathesachen 
\usepackage{amstext}
\usepackage[T1]{fontenc} % Ligaturen, richtige Umlaute im PDF
\usepackage[utf8]{inputenc}% UTF8-Kodierung für Umlaute usw
% Schriften
\usepackage[printonlyused]{acronym} %Notwendig für Abkürzungsverzeichnis
\usepackage{mathpazo} % Palatino für Mathemodus
\usepackage{setspace} % Zeilenabstand
\onehalfspacing % 1,5 Zeilen
% Schriften-Größen
\setkomafont{chapter}{\Large\rmfamily} % Überschrift der Ebene
\renewcommand*{\chapterheadstartvskip}{\vspace*{-20mm}}
\setkomafont{section}{\large\rmfamily}
\setkomafont{subsection}{\normalsize\rmfamily}
\setkomafont{subsubsection}{\normalsize\rmfamily}
\setkomafont{paragraph}{\normalsize\rmfamily}
\setkomafont{subparagraph}{\normalsize\rmfamily}
\setkomafont{chapterentry}{\normalsize\rmfamily} % Überschrift der Ebene in Inhaltsverzeichnis
\setkomafont{descriptionlabel}{\bfseries\rmfamily} % für description Umgebungen
\setkomafont{captionlabel}{\small\bfseries}
\setkomafont{caption}{\small}
% Sprache: Deutsch
\usepackage[ngerman]{babel} % Silbentrennung
% PDF
\usepackage[ngerman,pdfauthor={Markus Kättner},  pdfauthor={Markus Kättner}, pdftitle={Abschlussarbeit}, breaklinks=true,baseurl={}]{hyperref}
\usepackage[final]{microtype} % mikrotypographische Optimierungen
\usepackage{url}
\usepackage{pdflscape} % einzelne Seiten drehen können
% Tabellen
\usepackage{multirow} % Tabellen-Zellen über mehrere Zeilen
\usepackage{multicol} % mehre Spalten auf eine Seite
\usepackage{tabularx} % Für Tabellen mit vorgegeben Größen
\usepackage{longtable} % Tabellen über mehrere Seiten
\usepackage{booktabs} % schoenere Tabellen + abstaende
\usepackage{array}
%  Bibliographie
\usepackage{bibgerm} % Umlaute in BibTeX
% Tabellen
\usepackage{multirow} % Tabellen-Zellen über mehrere Zeilen
\usepackage{multicol} % mehre Spalten auf eine Seite
\usepackage{tabularx} % Für Tabellen mit vorgegeben Größen
\usepackage{array}
\usepackage{float}
\usepackage{titling}
\usepackage{enumitem} 
% Bilder
\usepackage{graphicx} % Bilder
\usepackage{color} % Farben
\graphicspath{{images/}}
\DeclareGraphicsExtensions{.pdf,.png,.jpg} % bevorzuge pdf-Dateien
\usepackage{subfig}
\usepackage[subfigure, titles]{tocloft}
\renewcommand{\cftchapdotsep}{\cftdotsep}
\renewcommand{\cftchapleader}{\cftdotfill{\cftchapdotsep}}
\newcommand{\subfigureautorefname}{\figurename} % um \autoref auch für subfigures benutzen
\usepackage[all]{hypcap} % Beim Klicken auf Links zum Bild und nicht zu Caption gehen
% Bildunterschrift
\setcapindent{0em} % kein Einrücken der Caption von Figures und Tabellen
\setcapwidth[c]{0.9\textwidth}
\setlength{\abovecaptionskip}{0.2cm} % Abstand der zwischen Bild- und Bildunterschrift
% Quellcode
\usepackage{listings} % für Formatierung in Quelltexten
\definecolor{grau}{gray}{0.25}
\lstset{
	extendedchars=true,
	basicstyle=\tiny\ttfamily,
	%basicstyle=\footnotesize\ttfamily,
	tabsize=2,
	keywordstyle=\textbf,
	commentstyle=\color{grau},
	stringstyle=\textit,
	numbers=left,
	numberstyle=\tiny,
	% für schönen Zeilenumbruch
	breakautoindent  = true,
	breakindent      = 2em,
	breaklines       = true,
	postbreak        = ,
	prebreak         = \raisebox{-.8ex}[0ex][0ex]{\Righttorque},
}
% linksbündige Fußboten
\deffootnote{1.5em}{1em}{\makebox[1.5em][l]{\thefootnotemark}}

\typearea{14} % typearea berechnet einen sinnvollen Satzspiegel (das heißt die Seitenränder) siehe auch http://www.ctan.org/pkg/typearea. Diese Berechnung befindet sich am Schluss, damit die Einstellungen oben berücksichtigt werden

%Überschriftennummerierung
\setcounter{secnumdepth}{4}
%Tiefe des Inhaltsverzeichnisses


\hypersetup{% 
  pdfborder= 0 0 0 
}

%lstset Formatierungen
\lstset{
  basicstyle=\fontsize{11}{13}\selectfont\ttfamily
}

%Formatierung von Abbildungs und Tabellenverzeichnis/caption
\renewcommand{\cftfigpresnum}{Abb. }
\renewcommand{\cfttabpresnum}{Tab. }

\renewcommand{\cftfigaftersnum}{:}
\renewcommand{\cfttabaftersnum}{:}

\setlength{\cftfignumwidth}{2cm}
\setlength{\cfttabnumwidth}{2cm}

\setlength{\cftfigindent}{0cm}
\setlength{\cfttabindent}{0cm}

\renewcaptionname{ngerman}{\figurename}{Abb.}
\renewcommand{\tablename}{Tab.}

%Natbib zum zitieren im Stil (Autor Jahr, Seite)
\usepackage{natbib}
\bibliographystyle{dcugerman}%GEÄNDERT:: alter wert: dcugerman
\usepackage{har2nat}
\providecommand\harvardand{}
\renewcommand{\harvardand}{und} 
\setcitestyle{aysep={},yysep={;}}

%Befehle für Multizitate
\newcommand{\doublecite}[2]{(#1; #2)}
\newcommand{\tripplecite}[3]{(#1; #2; #3)}
\newcommand{\quadrocite}[4]{(#1; #2; #3; #4)}
\newcommand{\quintcite}[5]{(#1; #2; #3; #4; #5)}

% fortlaufende nummerierung der footnotes
\usepackage{chngcntr}
\counterwithout{footnote}{chapter}

% einruecken des textes pro abschnitt
%\usepackage{parskip}
%\setlength{\parindent}{4mm}