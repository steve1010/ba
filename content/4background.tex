\chapter{Technical Background/Technology Assessment}

This section focuses on the technical background and theoretical foundations for I/O positioning systems. Advantages and disadvantages of single solutions are worked out to build a knowledge basis for the implementation of an I/O navigation prototype.
This prototype shall be able to navigate users at arbitrary locations outdoors to a predefined location indoors, in this case a room at the university campus. The transition between indoor and outdoor positioning shall be as smooth and user-friendly as possible.

To fulfil this task, the system needs several components such as at least one indoor and one outdoor positioning technology, localization methods, an underlying data model and a transition strategy for positioning technologies.
Representatives of those components are presented in the following and then evaluated at the end of this chapter wrt. their applicability. As this work focuses on the transition between outdoor and indoor navigation, for the prototype only one indoor and one outdoor technology is used. A combination of several technologies and methods would improve quality of positioning services \cite{challengesLBS}.
Further system design considerations and decisions are part of the next chapter. 


\section{Positioning Methods}

Localization methods can be used in combination with both indoor and outdoor technologies and form the algorithmical basis of every positioning system.
Although GNSSs which uniquely make use of the \ac{toa} method, are the most promising solutions for outdoor positioning, the application of other methods outdoors is also conceivable and partially already in operation.

There exist several methods for position determination, such as \textit{Proximity Detection}, \textit{Dead-Reckoning}, the time-based methods \textit{\ac{toa}} and \textit{\ac{rtof}}, also refered to as \textit{Trialateration}, \textit{Fingerprinting}, \textit{Map Matching} and \textit{Particle Filters}. 
Their applicability depends on environmental properties and the required positioning accuracy.

\textit{Hint: The content of the following section \emph{4.1 Positioning Methods} is extracted in reduced form from the author's last paper 'Indoor Localization - A Comparison of Different Methods and Approaches' \cite{posMethods} }

Like depicted in Figure \ref{img:methodsOverview}, there are three classes of methods: proximity detection, triangulation and scene analysis. 

\image{8cm}{grafiken/semImg/methodsOverview.PNG}{An overview to indoor localization methods \cite{recentAdvances}}{img:methodsOverview}

The later includes image- and video-based positioning, but also fingerprinting techniques that "first collect features (fingerprints) of a scene and then estimate	the location of an object by matching online measurements with the closest a priori location fingerprints." \cite{wirelessILSystemsAndTechniquesSurvey}  Related to scene analysis techniques are map matching and particle filters. Triangulation can either be performed with the angulation or lateration methods.
Proximity detection methods estimate a target's current position based on proximity to base stations or on movement patterns like \ac{dr}. 

% maybe split fingerprinting in scene analysis - video based, fingerpriting based ...

\subsection*{Proximity Detection}

The proximity detection method is a very simple localization method where multiple detection sensors (like routers, \ac{ir} or \ac{rfid}) are used to mark proximity areas. 

On detection, an object can be estimated to positioned in the detector's area. Whenever more than one sensor detects the same object, it is expected to be in the area of that sensor receiving the strongest signal. This process is also called \textit{forwarding}.
An illustration of this method can be seen in Figure \ref{img:proximityDetection}.
\image{5cm}{grafiken/semImg/proximityDetection.png}{The proximity detection method \cite{wirelessILSystemsAndTechniques}}{img:proximityDetection}

In case further or even global range is required, the cellular mobile \ac{gsm} network can be used to identify the radio station the device is allocated to (e.g. its cell-id). The observable is then expected to be in proximity to the radio station, hence in its cell. %\footnote{for more details on outdoor localization, see:\\ location determination using cell id: \url{{https://cell-id.info}} , and \\opencellid map:\\ \url{{https://www.opencellid.org/#zoom=16\&lat=49.8934\&lon=10.89114}}} 
\cite[p.5]{wirelessILSystemsAndTechniquesSurvey} For this reason, this method is also called cell of origin or cell identification.

For both, indoor and outdoor localization, the method's accuracy highly depends on the "density of beacon point deployment and signal range" \cite[p. 3]{wirelessILSystemsAndTechniques}. For improvement of accuracy, which is very limited in general, parameters like signal travel time can be included (see methods RSS, ToA and TDoA).

\subsection*{Triangulation}

This family of methods is based on calculations on the properties of triangles. For location determination, the known position and distance of three or more base stations is used to locate an object. There are basically two derivations of triangulation, namely lateration, which subsumes time- and distance-based methods, and angulation.

There are several varieties of triangulation methods, e.g. angle-based, time-based and signal property-based methods \cite{recentAdvances}. 

\subsubsection{Angulation - Angle-Based} 
In the angle, or \ac{aoa}-based localization method, intersections of at least two lines of bearing are computed. Although angulation with two base stations could work in some cases, it is mostly implemented with three base stations to improve accuracy - then also called triangulation. \cite{recentAdvances} reveals that for "finding direction, it requires highly directional antennas or antenna arrays", which forms, despite the fact that they are very costly, a big issue for indoor localization. Interferences origining from entities (people, walls, other signals) in indoor environments distort measurement values and result in erroneous position calculation. In (one of the rare) cases the signals remain uninterferenced, this technique would bring excellent accuracy.
Figure \ref{img:angleBased} shows basic operations with that method.
\image{7cm}{grafiken/semImg/Angulation.PNG}{The angle-based method\cite{recentAdvances}}{img:angleBased}

\subsubsection{Lateration - Time-Based}
Lateration methods calculate "a position determined from distance measurements" \cite{recentAdvances}, either relying on abolute or relative travel time. Unlike to angulation, lateration with two base stations does not bring any significance: like depiced in Figure \ref{img:otdoa}, two hyperbolic shapes (here simplified as cycles) always have two intersection points such that an exact position can not be determined. Thus, lateration methods are implemented with three or more base stations, then refered to as trilateration or multilateration. \\
All methods share the assumption that electromagnetic waves propagate with the same speed as light ($\sim$300.000 km/second).\\
There are four approaches:

\begin{itemize}
	\item \ac{toa}:\\
	This method requires all nodes in the network (object under observation and receiving stations) to be highly synchronized at any time. 
	The localization process starts at client side, where a timestamped signal is emitted and received by multiple base stations. The single distances d\textsubscript{i} are computed using absolute travel time t and the assumed propagation speed c: \\		 
	Assume 
	\[ t\textsubscript{i} = t\textsubscript{receive\textsubscript{i}} - t\textsubscript{send} \]
	with t\textsubscript{send} as sending and t\textsubscript{receive\textsubscript{i}} as time of arrival at receiver station i, then
	\[ d\textsubscript{i} =  t\textsubscript{i}  *  c \text{  .}\] 
	Observing the equation, it can be seen why precise synchronization is an absolute requirement for this method: due to constant c being that high, even small discrepancies in time can result in a completely different distance.\\
	Due to time-based signals are not directed in any way, an object must be located anywhere on a circle around the receiver station, with a radius of the computed distance. The intersection of those circles is then the most likeliest position.
	%The TOA method using three base stations is shown in Figure \ref{img:toa}.
%	\image{5cm}{grafiken/semImg/RToFandTOA.PNG}{The RToF/TOA method \cite{recentAdvances}}{img:toa}
	
	\item \ac{otdoa}:\\
	This time-based method, also called multilateration, infers positions by using relative arrival times captured by at three time-synchronized base stations with known location. A signal's OTDOA at the base stations can also be converted to a distance: \[ D = OTDOA * c\]
	For example, regarding two base stations BS1 and BS2, an OTDOA of +3.3 $\mu$s would indicate, that BS1 is 1000m closer than BS2. 
	There are infinitely many points where the transmitter is exactly D entities closer to the one base station than to another. Combining those points results in a hyperbolic shape. The intersection of multiple hyperbolic curves then represents the most likeliest position of the transmitter.	
	
	\item \ac{rtof}:\\
	Similary to ToA, the RToF method uses absolute time to localize objects. The only difference is that the RtoF method uses rount trip time, i.e. the time required to send a message from mobile transmitter to receiver station and back to transmitter, instead of only the time required to send and receive once. 
	In case interferences occur only in one of the two transmissions, averaging can be applied, such that this method can supply a more accurate positioning result than ToA. Obviously, the above kinds of lateration could also be combined with RtoF method in order to improve positioning accuracy. Also, positioning calculations could then be done on client-side.
	
	The lateration method is illustrated in Figure \ref{img:otdoa}, on which the bespoke hyperbolic curves are depicted in green.
	\image{7cm}{grafiken/semImg/otdoa.PNG}{The observed time difference of arrival method \cite{uwbILSpic}}{img:otdoa}
	
	\item \ac{tdoa}:\\
	\ac{tdoa} makes use of at least two different technologies, or rather their different propagation times, and relative arrival times. In many systems, electromagnetic and ultrasound waves are used, such as in \cite{otoaUltrasound}. A mobile client simultanously emmits signals using both technologies. At receiver side, the (relative) difference of arrival times and the actual phase difference are used to compute the distance. \cite{mitILSystem}
\end{itemize}

\subsubsection{Signal Property-Based - \ac{rssi}}

Besides regarding time as mayor indicator for distance, the received signal strength indicator (RSSI) approach focuses on a technology's propagation model. The distance to a sending point can then be estimated based on inverse-square law:

\[intensity \sim \frac{1}{distance^{2}}\]
Assuming an isotropic source (like APs or routers), the intensity of the emitted signal is inversely proportional to the square of the distance. Figure \ref{img:rssInverseSquare} illustrates this relation: in nearby areas, the intensity (illustrated with red arrows and red dots in the single areas), is significantly bigger than in marginal areas.	

\image{6cm}{grafiken/semImg/rssInverseSquare.PNG}{The inverse square law used in signal property-based methods\cite{imgSrc}}{img:rssInverseSquare}

As in other trilateration techniques, the single distances from a mobile client to the base stations form circles around them, whose intersection finally represents the positioning result.
%The signal property-based method with just a single base station is depicted in Figure \ref{img:rssMethod}.

%\image{5cm}{grafiken/semImg/rssMethod.PNG}{Signal property-based lateration \cite{recentAdvances}}{img:rssMethod}

%A related but more sophisticated approach is the so-called path loss model which is described by the following equation:
%\[RSSI = -10nlog_{10}(d)+C\]
%where "n is the path loss exponent that varies depending on the environment, d is the distance between the	transmitting and receiving devices, and C is a fixed constant that accounts for system losses" \cite{rssiIoT}. Including knowledge about a certain environment leads to a refinement of positioning results and  interference error reduction.

%An exemplary propagation model of BLE in indoor enviromnents can be seen in Figure \ref{img:rssiProp}.% if space needed: \\ after image declaration.
%\image{7cm}{grafiken/semImg/rssBLEpropModel.PNG}{Exemplary RSSI propagation model of BLE in indoor environments \cite{rssiIoT}}{img:rssiProp} 
%END
\subsection*{Dead Reckoning (DR)}

Another very simple but widely used method is \textit{Dead Reckoning}, where localization is "based on last determined	position and incrementing that position based on known or estimated speeds over elapsed time" \cite{recentAdvances}. The result can thus only result in an estimation of the current position. Furthermore, "inaccuracy of the process is cumulative, so the deviation in the position
fix grows with time" \cite{recentAdvances}. DR is also implemented in inertial navigation systems and is employed when GPS signal is lost, e.g. in tunnels.

For indoor purposes, according to \cite{drSystem}, accelerometer sensors in smartphones or attached sensors modules can be utilized to obtain the travelled distance, where distance estimation is then based on steps detected and the direction obtained by the magnetic sensor.
Besides this orientation projection, the overall localization process includes "filtering, step detection and step length estimation" \cite{drSystem}. As gravity and noise interferences accelerometer measurements, the values are filtered using physical constants and formulas, which will not be explained further here but can be looked up at \cite[p.3]{drSystem}.
Step detection can either be performed with zero-crossing or peak detection, whereas the later uses local maxima of the curve emerging from accelerometer measurement values. Unfiltered, all maximas were considered as steps, such that thresholds for further filtering are applied: all values under a certain threshold and also those not fitting in a typical time intervals of steps while walking (120ms – 400ms, according to \cite{drSystem}) are rejected.

An exemplary measurement result with three valid steps detected (depicted in blue) and the threshold (in red) is illustrated in Figure \ref{img:drMsmnt}.
\image{8cm}{grafiken/semImg/drMeasurement.PNG}{An exemplary DR measurement result with three steps detected \cite{drSystem}}{img:drMsmnt}

Zero-cross detection announces a step whenever the x-axis is crossed within a typical time interval (see peak detection). Obviously, this process is very errorprone, especially when step-size or -interval change.

Last step in the process is step length estimation. Besides some dynamic methods, which can be looked up in \cite{drSystem}, the static approach uses a person's height and a gender-specific constant: \[stepSize = height * k\] with "k equal to 0.415 for men and 0.413 for women." \cite{drSystem}
%	\image{5cm}{grafiken/semImg/drProcess.PNG}{The Dead Reckoning processes \cite{drSystem}}{img:dr}

\subsection*{Map Matching}

The Map Matching method makes use of a building's roadmap and (mostly noisy) localization samples, such that "building information provide a logical threshold to bound the solution into a certain region"\cite{mmSystem}. Thus, a system is able to return valid positions somewhere on the floor plan, even if the real measurement values diverge.

Map matching positioning samples mostly origin from WLAN signal strength values.
The required electronic map can either be designed in a floor plan builder or generated automatically using crowd-sourced data samples. 

There is a variety of applicable algorithms and data models such as graph- and link-based or room- and floor 3D-models. In the link-based approach, an ID and respective coordinates are assigned to each room. 
Map matching algorithms typically origin from the following types: 
1) geometrical, where the shortest distance from a measurement point to a trajectory is calculated, 2) topological, where links, i.e. the floors own attributes like "floor number, proximity of a height change access, stairs, and all possible links diverged from its start and destination nodes"\cite{mmSystem}, 3) probabilistic, where a target's direction and speed is considered (i.e. DR) and 4) sophisticated methods using fuzzy logic, belief theory and Bayesian networks. \\
Figure \ref{img:mmSolution}	shows a typical map matching navigation result with floorplan in black, measurement values in blue and map matched path in red.

\image{9cm}{grafiken/semImg/mmSolution.PNG}{A typical map matching navigation result \cite{mmSystem}}{img:mmSolution}

\subsection*{Fingerprinting}

Fingerprinting can be seen as an extended version of the RSSI method, where a database and further positioning algorithms are added to improve accuracy. 
The overall process is splitted into two parts, namely training (calibration, offline) and tracking (localization, online) phase.
The basic idea in the training phase is to "collect features of the scene (fingerprint) from the surrounding signatures at every location in the areas of interest and then build a fingerprint database" \cite{recentAdvances}. 
In the tracking phase, a mobile client requests all base stations in its region to reply to its positioning query. The strengths of signals received by the client can then be compared with those in the database, resulting in the closest known position.

% Example with explaination
A mobile device could for example collect WLAN signal strengths from different \ac{ap}s at every point of interest in a building and safe them together with actual location in a database. During online phase, current RSSI measurements can be mapped to known locations.\\
The overall process is illustrated in Figure \ref{img:fingerprinting}.

\image{8cm}{grafiken/semImg/fingerprinting.PNG}{The location fingerprinting process \cite{recentAdvances}}{img:fingerprinting}

\subsection*{Particel Filters}
Particel filter method represents a class of hybrid methods combining the advantages of DR and Map Matching with sequential Monte Carlo approximations.\cite{particleFilter}\\
The idea is to use randomly distributed samples - so-called \textit{particles} to represent all possible positions of an object. Based on any available information, e.g., pre-configured floorplans, motion profiles and laws of physics (e.g., targets move away with a well-known speed throgh buildings and can not go through walls, etc.), a \textit{weight} or probability is assigned to each particle. The next step is \textit{resampling}: Whereas unlikely particles are removed, those with enough weight are re-distributed according to their density and weight.
The resampled particle set is then relocated based on the motion model. In order to eliminate noise in the resulting set, the particles are again distributed among the now likeliest places (see step \emph{Diffuse} in Figure \ref{img:particleFilter}) and finally weighted based on available DR (and possibly additional) positioning information.
That location showing the highest probability density is then the particle filtering positioning result.\\

The overall particle filtering process including subprocesses is depicted in Figure \ref{img:particleFilter}.

\image{9cm}{grafiken/semImg/particleFilters.PNG}{The processes involved in particle filtering \cite{particleFilterImg}}{img:particleFilter}

\newpage


\section{Outdoor Positioning}
This section is about the outdoor positioning technologies GNSS, WiMAX and cellular networks. GNSSs are on the rise and only global solution, but other technologies are also capable to provide localization services in limited outdoor areas. Although their main drawback, limited coverage, makes them hardly applicable in a global context.
Besides that a globally networked installation fails due to its inapplicability in many places at earth (oceans, high mountains or deep valleys), also the required density and amount of base stations to reach comparable coverage and accuracy to GNSSs is technically and monetarily inconceivable.


\subsection*{\ac{gnss}}
% 0) quick Overview
% 1) Geschichte und Verbreitung
% 2) Anwendungen - wer nutzt es? 
% 3) Generelle Funktionsweise
% 4) Besondere Eigenschaften - Vor und Nachteile
%--------------------------------

% 0) quick Overview
Global Navigation Satellite Systems provide positioning functionalities with global range for navigation, emergency rescue and other applications of public and military use to land, air and water.

% 1) Geschichte und Verbreitung
%-------------------------------
As already briefly outlined in Section 1, the only fully operational and most popular \ac{gnss} is NAVSTAR (Navigational Satellite Timing and Ranging) \ac{gps}. It was developed by the U.S. DoD since 1973 until being fully operational in 1995. 
Besides GPS, there are various other \ac{gnss} approaches, such as China's Beidou (Kompass), India's IRNSS (Indian Regional Navigation Satellite System) Russia's Glonass and E.U.'s Galileo. \cite{heiseOnlineGPS}


% 2) Anwendungen - wer nutzt es? 
%--------------------------------
GPS is applied in social, economic and scientific areas. Typical applications range "from spacecraft navigation and geodesy, to land surveying and mapping, to precise agriculture and vehicle fleet management, to emergency services and professional navigation, to mass market applications such as in mobile devices (cars and smartphones) and location based services (LBS)."\cite{liRizos}

%Galileo(neuere, hochpräzise atomuhren und schnellere übertragungstechnologien als GPS- welche? stärkeres signal und 3 frequenzbänder)
%Rettungssyteme: rettungsignale empfangen und weiterleiten
%Galileo 2020 fully operationable 27 satellites 3 ersatz in 23.222 km mit 3.6km/s, 17 umläufe in 10 tagen danach wiederholung der umkreise
%2011 2 prototypsatelliten , dann 2 weitere im okt 2012 und steuerung am boden (IOV) (astrium gmbh) 
%kosten 7-10mrd euro. 


% 3) Generelle Funktionsweise
%-----------------------------
%Doppler Effekt: zeitliche Dehnung eines Signals entsprechend der Geschwindigkeit.
The common concept of all GNSSs are interconnected, clock-synchronized satellites (mostly more than 20) and ground stations such that trilateration (\ac{toa}) with a radio signals (GPS L1 signals: 1575,42 MHz) and calculations respecting the doppler effect are applicable. By that, an accuracy of 2-500m can be achieved.

%GPS Almanach: umlaufbahnen, sat status, uhrabweichungen, atmosphärische daten. 1 woche gültig-> muss runtergeladen werden (bis zu 12min über satellite, übers gsm(A-GPS) deutlich schneller)
General information about exact satellite orbits, statuses, clock deviations and atmospheric data is summarized under the GPS Almanach. Clients with corresponding GPS receiver can download this Almanach and start positioning. The Almanach is mostly receiced over the GSM network, as downloads directly from the satellite might take several minutes and is thus not applicable.

Whereby other trilateration methods calculate positions using distances to three base stations, GPS uses the Time-of-Arrival method which requires clock synchronization in all entities and thus four satellites: As the clocks on receivers like smartphones might differ form those on satellites, a fourth satellite is required to deal with that synchronization task. \cite[p. 58]{hybridizationGNSSPhd}

%wenn satelliten nahe beieinander liegen: schnittfläche groß, unpräziser.
%Dilution of precision: Geometric, horizontal, vertical, position(3d), time DOP
The geomatric position of used satellites is also an important factor: if satellites rely closed to each other, the intersection line is larger and positioning more imprecise (see ToA method section). Thus, in order to inform about such positioning deteriorations, every GPS signal carries the so-called dilutions of precision (DOP) which is calculated wrt. geometric, horizontal and vertical orientation. Values ranging from 1 to 6 indicate good precision, signals with DOP higher than 10 are basically not evaluable.

%Ablauf: Funksignal mit Uhrzeit und individualler code (satellitspezifisch)vom Satellit gesendet. GALILEO: 4 atomuhren (passiver wasserstoff-maser(1s/3mio jahre zeitabweichung) und ersatzweise rubidium-atomuhr(1s/760.000 Jahre) pro satellit, die permanent von der erde aus aktualisiert werden)
%299.000.000 m/s % Wattzahl wichtig?
Every satellite emitts a radio signal including timestamp and individual code, which can then be received by e.g. smartphones or car navigation systems. All positioning calculations are thus performed on client-side.
In order to be synchronized and precise, satellites are mostly equipped with multiple, frequently updated atomic clocks, such as Galileo satellites with two passive hydrogen maser clocks (deviation of 1s/3 mio years) and, alternatively, two rubidium atomic clocks (1s/760.000 years). 
%TODO: SNR und accuracy parameters

% 4) Besondere Eigenschaften - Vor und Nachteile
%-------------------------------------------------
% TODO: was macht eine minimale abweichung aus ?
% v=s/t | *sec -> s=v*t -> x = 300km/s * 0,00001s


%Störanfälle: zeit oben vergheht schneller als unten, Handy als empfänger, atmosphärische ablenkung/winkeländerung der Ionosphäre (lösung: 2 versch. frequenzempfänger oder nutzung von korrekturwerten , die am boden ermittelt und zum satelliten gesendet werden, so brauchen handys keinen teuren 2. empfänger), signalreflektion, (ungeplante)standortänderung  der vorausberechneten bahn von satelitten, 
Accuracy and functioning of GNSSs has also its limits and is susceptible to faults. Occuring issues are clock deviations, cheap (inaccurate) GNSS receivers on smartphones or navigation systems, ionospheric disturbances leading to angle change of signals, change of satellite constellation (less/other satellites available), signal reflection and multipath issues.

A Satellite-based augmentation systems (SBAS), such as the European Geostationary Nevigation Overlay Service (EGNOS) can be used to improve accuracy: reference stations deployed across the area of interest report all measured GNSS errors to base stations, where errors are collected, processed and send to geostationary satellites. The satellites then broadcast the augmentation information as overlay to the origninal GNSS message. \cite{egnos}
%TODO: GPS pic finden.. minor importancy
%- Dämpfung:\\
%Wohnhäuser: 5 bis 15 dB
%- Historische Gebäude: 25 bis 35 dB
%- Bürogebäude: 30 dB
%- Tiefgaragen: > 30 dB
% aus: Eissfeller/Teuber/Zucker, Indoor-GPS: Ist der Satellitenempfang in Gebäuden möglich?
Inside and around buildings GNSS signals are not or rarely available.
\cite{gpsIndoorsMoeglich} state that in houses there is an attenuation of 5-15dB (4-20 fold), in offices around 30dB (1000 fold) and in underground parking over 30dB.
They conclude that signal acquisition through concrete walls with 25dB attenuation and more is not possible without further assistence.

% Galileo Videos: https://www.youtube.com/watch?v=SkbP5nQnRZc und 
% Satellitennavigation: https://www.youtube.com/watch?v=lPkETIy0P9E

\subsection*{WiMAX}
% 0) quick Overview
% 1) Geschichte und Verbreitung
% 2) Anwendungen - wer nutzt es? 
% 3) Generelle Funktionsweise
% 4) Besondere Eigenschaften - Vor und Nachteile
%--------------------------------
% 0) quick Overview
This technology is part of the IEEE 802.16 protocol family and like WLAN (IEEE 802.11) a radio technology. Unlike WLAN, WiMAX (Worldwide Interoperability for Microwave Access) can operate at "higher speeds, over greater distances and for a greater number of users" \cite{wimax}.

% 1) Geschichte und Verbreitung

% 2) Anwendungen - wer nutzt es? 

% 3) Generelle Funktionsweise
In particular, this technology works with partially interconnected base stations which are connected to the internet via high bandwidth, wired connection and are wireless accessible by receivers within a cell radius of three to ten kilometers \cite{wimaxForumFAQ}.

The overall functioning is illustrated in Figure \ref{img:wimax}.

\image{9cm}{grafiken//4)howWiMAXworks.PNG}{How WiMAX works  \cite{wimax}}{img:wimax}

% 4) Besondere Eigenschaften - Vor und Nachteile
There are two modes of operation: Line-of-sight and non-line-of-sight, whereas the former sends at higher frequencies (licenced 2-11 GHz and unlicenced 10-66 GHz) and is thus more stable and offers lots more bandwidth \cite{wimax}.
By that, wireless internet access in rural areas and small cities can be provided, and can even deep indoors be accessed, e.g. with receiver stations on buildings acting like as repeaters.


\subsection*{Cellular Networks/GSM}
% 0) quick Overview
% 1) Geschichte und Verbreitung
% 2) Anwendungen - wer nutzt es? 
% 3) Generelle Funktionsweise
% 4) Besondere Eigenschaften - Vor und Nachteile
%--------------------------------

% 0) quick Overview
Cellular networks such as mobile \ac{gsm} networks for telecommunication are organized in cells of different size.
 % 3) Generelle Funktionsweise
In general, positioning in those cells is done with distance-based methods and is thus, due to the base stations' high distance, not very percise (25-100m whenever there is good coverage).
Cellular-based positioning is also possible in combination with triangulation and high-directional antennas.

% 4) Besondere Eigenschaften - Vor und Nachteile
Nevertheless, researchers found that accuracy with GSM could also be improved to up to 2.5m indoors using wide signal-strength fingerprints of up to "29 additional GSM channels, most of which are strong enough to be detected but too weak to be used for efficient communication." \cite{surveyWirelessIPS}


\section{Indoor Positioning}
This subsection presents various indoor positioning technologies and methods.
Whereas outdoor positioning typically relies GNSSs, indoors other technologies, such as WLAN, Bluetooth, RFID and Ultrasound can provide much higher accuracy.
% vllt noch system beispiele from +# a survey..

\subsection*{WLAN}
% 0) quick Overview
% 1) Geschichte und Verbreitung
\ac{wlan} (IEEE 802.11) was standardized in 1997 and is the most popular midrange local wireless networking technology. Its infrastructure, i.e.  routers and \ac{ap}s, is usually provided in most buildings.
WLAN devices operate at 11, 54, or 108 Mbps and have a typical range of 50-100m.
% 3) Generelle Funktionsweise
% 4) Besondere Eigenschaften - Vor und Nachteile

The most applied method with WLAN are signal-strength-based methods, such as RSSI and fingerprinting. Time- and distance-based methods are "less common in WLAN due to the complexity of time delay and angular measurements" \cite{wirelessILSystemsAndTechniques}.
The accuracy of positioning systems using RSS can vary from "3 to 30m, with an update rate of a few seconds" \cite{surveyWirelessIPS}. However, most systems can provide accuracy of less than 5m \cite{seamlessIOsolutions}, such as RADAR which was invented by a Microsoft research group in 2000 and provides 2-3m accuracy using \textquotedblleft signal strength and signal-to-noise ratio with
the triangulation location technique" \cite{surveyIPS}.

\subsection*{Bluetooth}
Bluetooth (IEEE 802.15) is a wireless \ac{wpan} standard operating at the 2.4 GHz ISM band. It offers "high security, low cost, low power and small size" \cite{recentAdvances} but has, compared to WLAN, also a smaller range of 10m to 15m and lower bit rates of 1 Mbps \cite{surveyWirelessIPS}.
Localization with Bluetooth works with small transceivers, called tags, which have unique IDs that can be used for localization.

The main drawback of Bluetooth-based positioning is the lengthy positioning process which requires \ac{bt} device discovery to be processed. As this step takes several seconds (10-30), it is unsuitable for real-time approaches \cite{recentAdvances}. Also pre-installation of tags in various points of interest is required. Nevertheless, indoor navigation with Bluetooth tags is a lightwight alternative to WLAN or Ultrasound.

\subsection*{RFID}
Localization with the \ac{rfid} technology works with wearable tags which emit or reflect radio waves. Scanning devices are organized in a network, whereas a single device can cover an area of several meters. RFID readers can read emitted data from tags which are available in two variants, active or passive. The former are battery-driven transceivers and are mostly applied in combination with proximity detection due to range limitations. But there are also tags which can transmit signals over tens of meters \cite{surveyWirelessIPS}, such there is basically any positioning method applicable. 
The RFID technology offers three different radio frequencies: low (125-134 kHz), high (13.56 MHz) and ultra-high frequency (860-960 MHz) and can thus (at lower frequencies) successfully operate at non-line-of-sight environments \cite{wirelessILSystemsAndTechniques}.

Passive tags act as replacement for traditional bar-code technology. They have no battery and can thus only "reflect the RF signal transmitted to them from a reader and add information by modulating the reflected signal" \cite{surveyWirelessIPS}.

The main drawback of the RFID technology is its limited range, the need for a tag to be carried along and the required installation of receivers.

\subsection*{Ultrasound}
Another positioning technology is Ultrasound, which is inspired by the orientation system of bats. It works with sound waves of frequencies higher than 16kHz which can not penetrate walls, i.e. positioning requires line-of-sight and obstacles can cause reflections.
Among advanced approaches, a wearable tag is still required. Ultrasound tags periodically emit signals which can be received at multiple stations.

Besides the remarkable positioning accuracy of several centimeters and a high sampling rate of 50 samples per second there is a dense network of sensors required and a single sensor can cover only a small area of approximately 1 m$^2$ \cite{surveyIPS}. The Active Bat positioning system, for example, which was invented by researchers at AT\&T Cambridge uses multilateration and 720 receivers fixed on the ceiling to cover an area of 1000 m$^2$. By that it is able to "determine the positions of up to 75 objects each second, accurate to around 3cm in three dimensions" \cite{activeBat}.

% evtl noch collision of metals als interference src
\section{Datamodels for Indoor Localization}
The data model is an essential part of transitional navigation systems, as it combines properties of both indoor and outdoor positioning. For example, GNSSs can only provide geometric location information, i.e. altitude and values in \ac{wgs} format (longitude, latitude). In contrast to that, localization inside buildings is oriented towards possibly hierarchical, symbolic cellspaces, i.e. rooms, corridors and floors. 
The model has also to consider navigation requirements such as distance, expected speed between locations, and the topological relations \textit{connected-to} and spatial containment.

There exist geometric and symbolic models.
A geometric model focuses on the shape of locations using points and polygons. 
Symbolic data models could be designed with set-based, hierarchical, graph-based or combined approaches. 

% \subsection*{Set-based}
In a \textit{set-based} model all symbolic locations are organized in a set. Special locations, such as all rooms in a floors, can be modelled as subsets.
In this approach, only qualitative distances can be used and there is no possibility to model relations, so it is not very useful for navigation.

% \subsection*{Hierarchical} 
\textit{Hierarchical} approaches model buildings hierarchically in sets according to the components' spatial relation. For example, buildings contain sets of floors and floors contains a sets of rooms.
There are still no real distances or connected-to relations included.

% \subsection*{Graph-based} 
\textit{Graph-based} models express connected-to relations between symbolic cellspaces with edges. They can also be weighted in order to express distances between locations. Based on the number of hops and their weight, a distance function and also range queries are applicable \cite{onLocationModels}.

% \subsection*{Combined}
\textit{Combining} different models improves capacity, for example adding range-based sets to the graph-based approach allows as well range queries as connected-to relations and calculation of distances.

As indicated before, for the prototype a hybrid location model is required which can handle both gemetric and symbolic locations. There are two approaches: the subspaces or the partial subspaces approach. Whereas the former assignes a geometric location to \textit{every} symbolic location, the later does so only for some locations, for example only for a building. The subspaces approach makes it possible to integrate symbolic cellspaces into geometric maps, whereby also geometric range queries are applicable \cite{onLocationModels} (e.g. query for restaurants or libraries within a radius of n kilometers).

\section{Indoor-/Outdoor Transition Solutions}

There are various strategies for switching between positioning technologies.
For reasons of simplicity, in this section GPS is assumed to be used for outdoor navigation and WLAN indoors. Of course the algorithms could also be applied to other techologies.

%\subsection*{vision/ user interaction-based approaches}
The most obvious and less technical strategy is to let the user decide. As pedestrians who navigate with smartphones through the city need to look at their smartphone anyway, the UI could also have interactive elements for I/O transition. For example, a hit on a respective button in the user interface (UI) could trigger a switching of technologies in the system.

\cite{streamspin} presents various strategies for I/O transitions.\\
% prefer GPS over WLAN
Two possible strategies are to always \textit{prefer GPS over WLAN} or vice versa.
The additional technology is only used if the main positioning technology detects no signal within a defined time interval (e.g. 5 seconds).
In case GPS is prefered, this leads to good outdoor positioning accuracy and an user's position could be detected successfully in the presence of windows. However, GPS positioning indoors is either not very good or not possible. It is thus more likely that a received GPS signal indoors interrupts reliable WLAN positioning. The other option, \textit{always prefer WLAN over GPS}, leads to good indoor accuracy but I/O transitions could not or way too late be detected, as WLAN signals can be received outdoors even tens of meters away from the buildings.\\
% Use GPS or WLAN until signal loss
Another strategy, \textit{prefer GPS or WLAN until signal loss}, aims to solve the problem of bad GPS signals indoors or bad WLAN signals outdoors. Although the transition between outdoors and indoors could be successfully detected due to GPS signal loss, going back outdoors would cause the same detection delay than the prefer-WLAN approach, which makes this strategy also not applicable for seamless I/O navigation.\\
According to \cite{streamspin}, the most promising strategy is to \textit{prefer GPS upon continuous readings}, which aims at solving the problem of bad WLAN positioning being further used outdoors. In this solution GPS is used when a signal is received every second for the last 5 seconds, if not, WLAN positioning is switched on.

%\subsection*{use SNR and GPS accuracy, dependent on going indoors or outdoors}
\cite{seamlessGPShandoverStrategy} propose a more advanced and very precise handover strategy. They found that while the "average SNR dropped from 30 dB to lower than 20 dB [..](,) satellites at an elevation of 30–90 degrees showed a dip in SNR changes [..] (while others) showed no noticeable SNR drop" \cite[p. 6]{seamlessGPShandoverStrategy}. That means the use of an average value of all received satellites could impair transition detection. The authors thus recommend to use only satellites of high altitude for outdoor-indoor transitions.
Furthermore, the authors found that SNR values are not adequate for handover from indoor to outdoor, as the SNR indoors in presence of windows or glass roofs is comparable to the SNR outdoors. In fact, their experiments show that GPS accuracy is a better indicator for indoor-outdoor transition: accuracy indoors varies from 20-50m, even in presence of windows or glass roofs, whereas accuracy outdoors is around 10m.
The comparison of SNR and GPS accuracy in indoor places with outdoors is shown in Figure \ref{img:snrComp}.
\image{16cm}{grafiken//snrComparison.PNG}{Comparison of (a) SNR and (b) GPS accuracy in the indoor places with those in outdoor environment. \cite{seamlessGPShandoverStrategy}}{img:snrComp}

\section{Assessment and Conclusion}

This chapter presented various methods, technologies, data models and transition solutions. Recent works \tripplecite{\cite{wirelessILSystemsAndTechniques}}{\cite{recentAdvances}}{\cite{seamlessGPShandoverStrategy}} already evaluated the bespoke components. The results are rehashed in the following.


\subsection*{Methods}
% proximity, triangulation, lateration (tdoa rtof ordoa, toa), rssi, dr, mm, fingerprinting, particle filters.

% \cite{wirelessILSystemsAndTechniques} methods comparison
\cite{recentAdvances} evaluated the bespoke methods under the parameters accuracy, coverage, line of sight (LOS) requirement, multipath affection and cost. The evaluation table is shown in Figure \ref{img:methodsEval}.

\image{12cm}{grafiken//semImg//methodsComparisonOriginal.PNG}{Evaluation of positioning methods wrt. accuracy, coverage, multipath affection and cost \cite{recentAdvances}}{img:methodsEval}

Angle-based and distance-based methods require LOS, which is typically not given inside buildings. However, in lofts and big halls those methods are applicable but interconnected with high costs.\\
% From fingerprinting: 
An advantage of fingerprinting is its low cost: it does "not require specialized hardware in either the mobile device or the receiving end nor is no time synchronization necessary between the stations." \cite{recentAdvances} On the other hand, the method has its drawbacks in the heavy calibration process: "In order to eliminate the deviation of attenuation in the signal, RSS values are to be averaged over a certain time interval up to several minutes at each fingerprint location" \cite{recentAdvances}. Infrastructural changes, for instance (re-) moving of objects and \ac{ap}s entail modifications in the RSSI database, such that the calibration process must be repeated periodically to keep the system up-to-date.\\
Advanced methods, such as particle filters, Kalman filters and neural networks are complex to setup and aim at improving accuracy more and more, which is basicall not required for simple navigation tasks.
In contrast to them, Dead-Reckoning can provide coverage of a large building even with few sensors. In case cost or simplicity is required this method fits best.

\subsection*{Outdoor positioning}
% Saltelliten, da z.b. sendemasten zu kleine gebiete abdecken können und außerdem in tälern (schattenseite) nicht positionieren
For outdoor positioning, a network of satellites is the only way to cover global range, as for example networked base stations would require partially infeasible installation of transceivers anywhere on earth (e.g. in oceans, mountains, dells and gorges etc.) to provide equal coverage. Local I/O navigation systems of limited range could also be implemented with other solutions such as WiMAX or cellular networks

\subsection*{Indoor positioning}
The choice of appropriate indoor positioning technology/technologies depend/s on the project's requirements, existing digital infrastructure and resources. If high accuracy is required, Ultrasound positioning might be the right choice. If costs are to be safed, existing infrastructure (WLAN) and smartphones or cheap RFID tags can be used. Compared to BT, RFID positioning is faster and thus more applicable for real-time applications. \cite{recentAdvances} evaluated the bespoke indoor postioning technologies wrt. the parameters accracy, coverage, power consumption and cost. The table is shown in Figure \ref{img:techEval}.

\image{12cm}{grafiken//technologiesComparisonCut.PNG}{Evaluation of positioning technologies wrt. accuracy, coverage, power consumption and cost \cite{recentAdvances}}{img:techEval}

\subsection*{Datamodel}
For an I/O navigation system obviously a hybrid (geometric and symbolic) model is required. An evaluation of symbolic models can be found in Figure \ref{img:modelsEval}. Graph-based or combined symbolic models are the most promising for navigation due to the support of distance and connected-to relation.	The implementation of the final data model is presented in the following chapter.

\image{10cm}{grafiken//propertiesLocationModels.PNG}{Evaluation of symbolic location models\cite{onLocationModels}}{img:modelsEval}

\subsection*{Transition solution}

The most reliable transition solution is to ask the users, as they can basically not fail. 
As already mentioned in the respective section, the only applicable technical strategies are \textit{prefer GPS upon continuous readings} or \cite{seamlessGPShandoverStrategy}'s approach with SNR drop of satellites at elevation of 30-90 degrees for outdoor/indoor transition and significant GPS accuracy rise for indoor/outdoor transition.

Also a combination of different approaches is conceivable, e.g. a navigation system could ask the user whether the detected transition is valid.

% END BACKGROUND % 