\chapter{Technical Background/Technology Assessment}

This section focuses on the technical background and theoretical foundations for I/O positioning systems. Advantages and disadvantages of single solutions are worked out to build a knowledge basis for the implementation of an I/O navigation prototype.
This prototype shall be able to navigate users at an an address arbitrary location outdoors to a predefined location indoors, in this case a room at the university campus.

To fulfill this task the system needs several components, such as at least one indoor and one outdoor positioning technology, localization methods, an underlying data model and a transition strategy for positioning technologies.
Representatives of those components are presented in the following and then evaluated at the end of this chapter wrt. their applicability. A combination of several technologies and methods would improve quality of positioning services \cite{challengesLBS}. Further system design considerations and decisions are part of the next chapter. 


\section{Positioning Methods}

Localization methods can be used in combination with both indoor and outdoor technologies and form the algorithmical basis of every positioning system.
Although GNSSs which uniquely make use of the \ac{toa} method, are the most promising solutions for outdoor positioning, the application of other methods outdoors is also conceivable and partially already in operation.

There exist several methods for position determination, such as \textit{Proximity Detection}, \textit{Dead-Reckoning}, the time-based methods \textit{\ac{toa}} and \textit{\ac{rtof}}, also refered to as \textit{Trialateration}, \textit{Fingerprinting}, \textit{Map Matching} and \textit{Particle Filters}. 
Their applicability depends on environmental properties and the required positioning accuracy.

\textit{Hint: The following content of this section \emph{4.1 Positioning Methods} is extracted from the author's last paper "Indoor Localization - A Comparison of Different Methods and Approaches"}

Like depicted in Figure \ref{img:methodsOverview}, there are three classes of methods: proximity detection, triangulation and scene analysis. 

\image{8cm}{grafiken/semImg/methodsOverview.PNG}{An overview to indoor localization methods \cite{recentAdvances}}{img:methodsOverview}

The later includes image- and video-based positioning, but also fingerprinting techniques that "first collect features (fingerprints) of a scene and then estimate	the location of an object by matching online measurements with the closest a priori location fingerprints." \cite{wirelessILSystemsAndTechniquesSurvey}  Related to scene analysis techniques are map matching and particel filters. Triangulation can either be performed with the angulation or lateration methods.
Proximity detection methods estimate a target's current position based on proximity to base stations or on movement patterns like \ac{dr}. 

% maybe split fingerprinting in scene analysis - video based, fingerpriting based ...

\subsection{Proximity Detection}

The proximity detection method is a very simple localization method where multiple detection sensors (like routers, \ac{ir} or \ac{rfid}) are used to mark proximity areas. 

On detection, an object can be estimated to positioned in the detector's area. Whenever more than one sensor detects the same object, it is expected to be in the area of that sensor receiving the strongest signal. This process is also called 'forwarding'.
An illustration of this method can be seen in Figure \ref{img:proximityDetection}.
\image{5cm}{grafiken/semImg/proximityDetection.png}{The proximity detection method \cite{wirelessILSystemsAndTechniques}}{img:proximityDetection}

In case further or even global range is required, the cellular mobile \ac{gsm} network can be used to identify the radio station the device is allocated to (e.g. its cell-id). The observable is then expected to be in proximity to the radio station, hence in its cell. %\footnote{for more details on outdoor localization, see:\\ location determination using cell id: \url{{https://cell-id.info}} , and \\opencellid map:\\ \url{{https://www.opencellid.org/#zoom=16\&lat=49.8934\&lon=10.89114}}} 
\cite[p.5]{wirelessILSystemsAndTechniquesSurvey} For this reason, this method is also called cell of origin or cell identification.

For both, indoor and outdoor localization, the method's accuracy highly depends on the "density of beacon point deployment and signal range" \cite[p. 3]{wirelessILSystemsAndTechniques}. For improvement of accuracy, which is very limited in general, parameters like signal travel time can be included (see methods RSS, ToA and TDoA).

\subsection{Triangulation}

This family of methods is based on calculations on the properties of triangles. For location determination, the known position and distance of mostly three base stations is used to locate an object. There are basically two derivations of triangulation, namely lateration, which subsumes time- and distance-based methods, and angulation, whereas both are sometimes (erroneously) refered to as triangulation.

There are several varieties of triangulation methods, e.g. angle-based, time-based and signal property-based methods\cite{recentAdvances}. All methods share the assumption that electromagnetic waves propagate with the same speed as light ($\sim$300.000 km/second).\\

\subsubsection{Angulation - Angle-Based} 
In the angle, or \ac{aoa}-based localization method, intersections of at least two lines of bearing are computed. Although angulation with two base stations could work in some cases, it is mostly implemented with three base stations to improve accuracy - then also called triangulation. \cite{recentAdvances} reveals that for "finding direction, it requires highly directional antennas or antenna arrays", which forms, despite the fact that they are very costly, a big issue for indoor localization. Interferences origining from entities (people, walls, other signals) in indoor environments distort measurement values and result in erroneous position calculation. In (one of the rare) cases the signals remain uninterferenced, this technique would bring excellent accuracy.
Figure \ref{img:angleBased} shows basic operations with that method.
\image{7cm}{grafiken/semImg/Angulation.PNG}{The angle-based method\cite{recentAdvances}}{img:angleBased}

\subsubsection{Lateration - Time-Based}
Lateration methods calculate "a position determined from distance measurements" \cite{recentAdvances}, either relying on abolute or relative travel time. Unlike to angulation, lateration with two base stations does not bring any significance: like depiced in Figure \ref{img:otdoa}, two hyperbolic shapes (here simplified as cycles) always have two intersection points such that an exact position can not be determined. Thus, lateration methods are implemented with three or more base stations, then refered to as trilateration or multilateration. \\

There are four approaches:

\begin{itemize}
	\item \ac{toa}:\\
	This method requires all nodes in the network (detected object and receiving stations) to be highly synchronized at any time. 
	The localization process starts at client side, where a timestamped signal is emitted and received by multiple base stations. The single distances d\textsubscript{i} are computed using absolute travel time t and the assumed propagation speed c: \\		 
	Assume 
	\[ t\textsubscript{i} = t\textsubscript{receive\textsubscript{i}} - t\textsubscript{send} \]
	with t\textsubscript{send} as sending and t\textsubscript{receive\textsubscript{i}} as time of arrival at receiver station i, then
	\[ d\textsubscript{i} =  t\textsubscript{i}  *  c \text{  .}\] 
	Observing the equation, it can be seen why precise synchronization is an absolute requirement for this method: due to constant c being that high, even small discrepancies in time can result in a completely different distance.\\
	Due to time-based signals are not directed in any way, an object must be located anywhere on a circle around the receiver station, with a radius of the computed distance. The intersection of those circles is then the most likeliest position.
	The TOA method using three base stations is shown in Figure \ref{img:toa}.
	\image{5cm}{grafiken/semImg/RToFandTOA.PNG}{The RToF/TOA method \cite{recentAdvances}}{img:toa}\\
	
	\item \ac{otdoa}:\\
	This time-based method, also called multilateration, infers positions by using relative arrival times captured by at three time-synchronized base stations with known location. A signal's OTDOA at the base stations can also be converted to a distance: \[ D = OTDOA * c\]
	For example, regarding two base stations BS1 and BS2, an OTDOA of +3.3 $\mu$s would indicate, that BS1 is 1000m closer than BS2. 
	There are infinitely many points where the transmitter is exactly D entities closer to the one base station than to another. Combining those points results in a hyperbolic shape. The intersection of multiple hyperbolic curves then represents the most likeliest position of the transmitter.
	The OTDOA method is illustrated in Figure \ref{img:otdoa}, on which the bespoke hyperbolic curves are depicted in green.
	\image{7cm}{grafiken/semImg/otdoa.PNG}{The observed time difference of arrival method\cite{uwbILSpic}}{img:otdoa}
	
	\item \ac{tdoa}:\\
	\ac{tdoa} makes use of at least two different technologies, or rather their different propagation times, and relative arrival times. In many systems, electromagnetic and ultrasound waves are used, such as in \cite{otoaUltrasound}. A mobile client simultanously emmits signals using both technologies. At receiver side, the (relative) difference of arrival times and the actual phase difference are used to compute the distance. \cite{mitILSystem}
	% TODO: IL-methods-TDoA: here could be more content 
	
	\item \ac{rtof}:\\
	Similary to ToA, the RToF method uses absolute time to localize objects. The only difference is that the RtoF method uses rount trip time, i.e. the time required to send a message from mobile transmitter to receiver station and back to transmitter, instead of only the time required to send and receive once. 
	In case interferences occur only in one of the two transmissions, averaging can be applied, such that this method can supply a more accurate positioning result than ToA. Obviously, the above kinds of lateration could also be combined with RtoF method in order to improve positioning accuracy. Also, positioning calculations could then be done on client-side. The method is depicted in Figure \ref{img:toa}.\\
\end{itemize}

\subsubsection{Signal Property-Based - \ac{rssi}}

Besides regarding time as mayor indicator for distance, the received signal strength indicator (RSSI) approach focuses on a technology's propagation model. The distance to a sending point can then be estimated based on inverse-square law:

\[intensity \sim \frac{1}{distance^{2}}\]
Assuming an isotropic source (like APs or routers), the intensity of the emitted signal is inversely proportional to the square of the distance. Figure \ref{img:rssInverseSquare} illustrates this relation: in nearby areas, the intensity (illustrated with red arrows and red dots in the single areas), is significantly bigger than in marginal areas.	

\image{8cm}{grafiken/semImg/rssInverseSquare.PNG}{The inverse square law used in signal property-based methods\cite{imgSrc}}{img:rssInverseSquare}

As in other trilateration techniques, the single distances from a mobile client to the base stations form circles around them, whose intersection finally represents the positioning result.
The signal property-based method with just a single base station is depicted in Figure \ref{img:rssMethod}.

\image{6cm}{grafiken/semImg/rssMethod.PNG}{Signal property-based lateration \cite{recentAdvances}}{img:rssMethod}

A related but more sophisticated approach is the so-called path loss model which is described by the following equation:
\[RSSI = -10nlog_{10}(d)+C\]
where "n is the path loss exponent that varies depending on the environment, d is the distance between the	transmitting and receiving devices, and C is a fixed constant that accounts for system losses." \cite{rssiIoT} Including knowledge about a certain environment leads to a refinement of positioning results and  interference error reduction.

An exemplary propagation model of BLE in indoor enviromnents can be seen in Figure \ref{img:rssiProp}.% if space needed: \\ after image declaration.
\image{7cm}{grafiken/semImg/rssBLEpropModel.PNG}{Exemplary RSSI propagation model of BLE in indoor environments \cite{rssiIoT}}{img:rssiProp} 
%END
\subsection{Dead Reckoning (DR)}

Another very simple but widely used method is Dead Reckoning, where localization is "based on last determined	position and incrementing that position based on known or estimated speeds over elapsed time" \cite{recentAdvances}, and thus can only result in an estimation of the current position. Furthermore, "inaccuracy of the process is cumulative, so the deviation in the position
fix grows with time". \cite{recentAdvances} DR is also implemented in ineartial navigation systems and is employed when GPS signal is lost, for example in tunnels.

For indoor purposes, according to \cite{drSystem}, accelerometer sensors in smartphones or even attached sensors modules can be utilized to obtain the travelled distance, where distance estimation is then based on steps detected and values obtained by the magnetic sensor, i.e., the compass direction.  (also called orientation projection).

Besides orientation projection, the overall localization process includes "filtering, step detection and step length estimation" \cite{drSystem}. As gravity and noise interferences accelerometer measurements, the values are filtered using physical constants and formulas, which will not be explained further here but can be looked up at \cite[p.3]{drSystem}.
Next in the DR process is step detection, which can either be done with peak- or zero-crossing detection. Peak detection uses local maximas of the curve emerging from accelerometer measurement values. Unfiltered, all maximas were considered as steps, such that thresholds for further filtering are applied: all values under a certain threshold and also those not fitting in a typical time intervals of steps while walking (120ms – 400ms, according to \cite{drSystem}) are rejected.

An exemplary measurement result with three valid steps detected (depicted in blue) and the threshold (in red) is illustrated in Figure \ref{img:drMsmnt}.
\image{8cm}{grafiken/semImg/drMeasurement.PNG}{An exemplary DR measurement result with three steps detected \cite{drSystem}}{img:drMsmnt}

Zero-cross detection announces a step whenever the x-axis is crossed within a typical time interval (see peak detection). Obviously, this process is very errorprone, especially when step-size or -interval change.

Last step in the process is step length estimation. Besides some dynamic methods, which can be looked up in \cite{drSystem}, the static approach uses a person's height and a gender-specific constant: \[stepSize = height * k\] with "k equal to 0.415 for men and 0.413 for women." \cite{drSystem}
%	\image{5cm}{grafiken/semImg/drProcess.PNG}{The Dead Reckoning processes \cite{drSystem}}{img:dr}

\subsection{Map Matching}

The Map Matching method makes use of a building's roadmap and (mostly noisy) localization samples, such that "building information provide a logical threshold to bound the solution into a certain region"\cite{mmSystem}. Thus, a system is able to return valid positions somewhere on the floorplan, even if the real measurement values diverge.

As nowadays most big buildings are equiped with many rounters or \ac{ap}s, map matching positioning samples mostly origin from WLAN.

The required electronic map can either be designed in a floor plan builder or generated automatically using crowd-sourced data samples. 

There is a variety of algorithms and applied datamodels such as graph- and link-based or room- and floor 3D-models. In the link-based approach, an ID and respective coordinates are assigned to each room. 
Map matching algorithms typically origin from the following types: 
1) geometrical, where the shortest distance from a mesaurement point to a trajectory is calculated, 2) topological, where links, i.e. the floors own attributes like "floor number, proximity of a height change access, stairs, and all possible links diverged from its start and destination nodes"\cite{mmSystem}, 3) probabilistic, where a target's direction and speed is considered (i.e. DR) and 4) sophisticated methods using fuzzy logic, belief theory and Bayesian networks.

Obviously, combinations of those algorithms improve positioning accuracy, such that they often occur at least pairwise. \cite{mmSystem}, for example, uses a combination of geometrical, topological and probabilistic algorithms to ensure an optimal result. The authors also refined the geometrical algorithm such that the shift between positioning estimate and projection is applied to the next measurement, which leads to a more accurate localization.

Figure \ref{img:mmSolution}	shows a typical map matching navigation result with floorplan in black, measurement values in blue and map matched path in red.

\image{9cm}{grafiken/semImg/mmSolution.PNG}{A typical map matching navigation result \cite{mmSystem}}{img:mmSolution}

\subsection{Fingerprinting}

Fingerprinting can be seen as an extended version of the RSS method, where a database and further positioning algorithms are added to improve accuracy. 
The overall process is splitted into two parts, namely training (calibration, offline) and tracking (localization, online) phase.

The basic idea in the training phase is to "collect features of the scene (fingerprint) from the surrounding signatures at every location in the areas of interest and then build a fingerprint database." \cite{recentAdvances} 

In the tracking phase, a mobile client requests all base stations in its region to reply to its positioning query. The strengths of signals received by the client can then be compared with those in the database, resulting in the closest known position.

% Example with explaination
For example, a mobile device could collect WLAN signal strengths from different \ac{ap}s at every point of interest in a building, safe them together with actual location in a database. During online phase, current RSS measurements can be mapped to known locations.

The overall process is illustrated in Figure \ref{img:fingerprinting}.

\image{8cm}{grafiken/semImg/fingerprinting.PNG}{The location fingerprinting process \cite{recentAdvances}}{img:fingerprinting}

Multiple algorithms can be chosen to perform the required comparing of live tracks with DB values, such as probabilistic methods and k-nearest-neighbor (kNN) averaging, but also advanced ones like neural networks, support vector machine (SVM), and smallest M-vertex polygon (SMP) like mentioned in \cite[p. 4f]{wirelessILSystemsAndTechniquesSurvey} 

An advantage of this method is its low cost: it does "not require specialized hardware in either the mobile device or the receiving end nor is no time synchronization necessary between the stations." \cite{recentAdvances} On the other hand, the method has its drawbacks in the heavy calibration process: "In order to eliminate the deviation of attenuation in the	signal, the RSS values are to be averaged over a certain time interval up to several minutes at each fingerprint location." \cite{recentAdvances} Infrastructural changes, for instance  (re-) moving of objects and \ac{ap}s entail modifications in the RSSI database, such that the calibration process must be repeated periodically to keep the system up-to-date.



\subsection{Particel Filters}
Particel filter method represents a class of hybrid methods combining the advantages of DR and Map Matching with sequential Monte Carlo approximations.\cite{particleFilter}\\

The idea is to use randomly distributed samples - so-called \textit{particles} to represent all possible positions of an object. Based on any available information, e.g., pre-configured floorplans, motion profiles and laws of physics (e.g., targets move away with a well-known speed throgh buildings and can not go through walls, etc.), a \textit{weight} or probability is assigned to each particle. The next step is \textit{resampling}: Whereas unlikely particles are removed, those with enough weight are re-distributed according to their density and weight.
The resampled particle set is then relocated based on the motion model. In order to eliminate noise in the resulting set, the particles are again distributed among the now likeliest places (see step \emph{Diffuse} in Figure \ref{img:particleFilter}) and finally weighted based on available DR (and possibly additional) positioning information.
That location showing the highest probability density is then the particle filtering positioning result.\\

The overall particle filtering process including subprocesses is depicted in Figure \ref{img:particleFilter}.

\image{9cm}{grafiken/semImg/particleFilters.PNG}{The processes involved in particle filtering \cite{particleFilterImg}}{img:particleFilter}




\section{Outdoor Positioning}
In this section, various outdoor positioning technologies (GNSS,WiMAX, GSM) are presented and then analyzed wrt. their applicability and accuracy.

\subsection*{\ac{gnss}}
% 0) quick Overview
% 1) Geschichte und Verbreitung
% 2) Anwendungen - wer nutzt es? 
% 3) Generelle Funktionsweise
% 4) Besondere Eigenschaften - Vor und Nachteile
%--------------------------------

% 0) quick Overview
Global Navigation Satellite Systems provide positioning functionalities with global range for navigation, emergency rescue and other applications of public and military use to land, air and water.

% 1) Geschichte und Verbreitung
%-------------------------------
As already briefly outlined in Section 1, the only fully operational and most popular \ac{gnss} is NAVSTAR (Navigational Satellite Timing and Ranging) \ac{gps}. It was developed by the U.S. DoD since 1973 until being fully operational in 1995. 
Besides GPS, there are various other \ac{gnss} approaches, such as China's Beidou (Kompass), India's IRNSS (Indian Regional Navigation Satellite System) Russia's Glonass and E.U.'s Galileo. \cite{heiseOnlineGPS}


% 2) Anwendungen - wer nutzt es? 
%--------------------------------
GPS is applied in social, economic and scientific areas. Typical applications range "from spacecraft navigation and geodesy, to land surveying and mapping, to precise agriculture and vehicle fleet management, to emergency services and professional navigation, to mass market applications such as in mobile devices (cars and smartphones) and location based services (LBS)."\cite{liRizos}

%Galileo(neuere, hochpräzise atomuhren und schnellere übertragungstechnologien als GPS- welche? stärkeres signal und 3 frequenzbänder)
%Rettungssyteme: rettungsignale empfangen und weiterleiten
%Galileo 2020 fully operationable 27 satellites 3 ersatz in 23.222 km mit 3.6km/s, 17 umläufe in 10 tagen danach wiederholung der umkreise
%2011 2 prototypsatelliten , dann 2 weitere im okt 2012 und steuerung am boden (IOV) (astrium gmbh) 
%kosten 7-10mrd euro. 


% 3) Generelle Funktionsweise
%-----------------------------
%Doppler Effekt: zeitliche Dehnung eines Signals entsprechend der Geschwindigkeit.
The common concept of all GNSSs are interconnected, clock-synchronized satellites and ground stations such that trilateration (\ac{toa}) with a radio signals (GPS L1 signals: 1575,42 MHz) and calculations respecting the doppler effect are applicable. By that, an accuracy of 2-500m can be achieved.

%GPS Almanach: umlaufbahnen, sat status, uhrabweichungen, atmosphärische daten. 1 woche gültig-> muss runtergeladen werden (bis zu 12min über satellite, übers gsm(A-GPS) deutlich schneller)
General information about exact satellite orbits, statuses, clock deviations and atmospheric data is summarized under the GPS Almanach. Clients with corresponding GPS receiver can download this Almanach and start positioning. The Almanach is mostly receiced over the GSM network, as downloads directly from the satellite might take several minutes and is thus not applicable.

Whereby other trilateration methods calculate positions using distances to three base stations, GPS uses the Time-of-Arrival method which requires clock synchronization in all entities and thus four satellites: As the clocks on receivers like smartphones might differ form those on satellites, a fourth satellite is required to deal with that synchronization task. \cite[p. 58]{hybridizationGNSSPhd}

%wenn satelliten nahe beieinander liegen: schnittfläche groß, unpräziser.
%Dilution of precision: Geometric, horizontal, vertical, position(3d), time DOP
The geomatric position of used satellites is also an important factor: if satellites rely closed to each other, the intersection line is larger and positioning more imprecise (see ToA method section). Thus, in order to inform about such positioning deteriorations, every GPS signal carries the so-called dilutions of precision (DOP) which is calculated wrt. geometric, horizontal and vertical orientation. Values ranging from 1 to 6 indicate good precision, signals with DOP higher than 10 are basically not evaluable.

%Ablauf: Funksignal mit Uhrzeit und individualler code (satellitspezifisch)vom Satellit gesendet. GALILEO: 4 atomuhren (passiver wasserstoff-maser(1s/3mio jahre zeitabweichung) und ersatzweise rubidium-atomuhr(1s/760.000 Jahre) pro satellit, die permanent von der erde aus aktualisiert werden)
%299.000.000 m/s % Wattzahl wichtig?
Every satellite emitts a radio signal including timestamp and individual code, which can then be received by e.g. smartphones or car navigation systems. All positioning calculations are thus performed on client-side.
In order to be synchronized and precise, satellites are mostly equipped with multiple, frequently updated atomic clocks, such as Galileo satellites with two passive hydrogen maser clocks (deviation of 1s/3 mio years) and, alternatively, two rubidium atomic clocks (1s/760.000 years). 
%TODO: SNR und accuracy parameters

% 4) Besondere Eigenschaften - Vor und Nachteile
%-------------------------------------------------
% TODO: was macht eine minimale abweichung aus ?
% v=s/t | *sec -> s=v*t -> x = 300km/s * 0,00001s


%Störanfälle: zeit oben vergheht schneller als unten, Handy als empfänger, atmosphärische ablenkung/winkeländerung der Ionosphäre (lösung: 2 versch. frequenzempfänger oder nutzung von korrekturwerten , die am boden ermittelt und zum satelliten gesendet werden, so brauchen handys keinen teuren 2. empfänger), signalreflektion, (ungeplante)standortänderung  der vorausberechneten bahn von satelitten, 
Accuracy and functioning of GNSSs has also its limits and is susceptible to faults. Occuring issues are clock deviations, cheap (inaccurate) GNSS receivers on smartphones or navigation systems, ionospheric disturbances leading to angle change of signals, change of satellite constellation (less/other satellites available), signal reflection and multipath issues.

A Satellite-based augmentation systems (SBAS), such as the European Geostationary Nevigation Overlay Service (EGNOS) can be used to improve accuracy: reference stations deployed across the area of interest report all measured GNSS errors to base stations, where errors are collected, processed and send to geostationary satellites. The satellites then broadcast the augmentation information as overlay to the origninal GNSS message. \cite{egnos}
%TODO: GPS pic finden.. minor importancy
%- Dämpfung:\\
%Wohnhäuser: 5 bis 15 dB
%- Historische Gebäude: 25 bis 35 dB
%- Bürogebäude: 30 dB
%- Tiefgaragen: > 30 dB
% aus: Eissfeller/Teuber/Zucker, Indoor-GPS: Ist der Satellitenempfang in Gebäuden möglich?
Inside and around buildings GNSS signals are not or rarely available.
\cite{gpsIndoorsMoeglich} state that in houses there is an attenuation of 5-15dB (4-20 fold), in offices around 30dB (1000 fold) and in underground parking over 30dB.
They conclude that signal acquisition through concrete walls with 25dB attenuation and more is not possible without further assistence.

% Galileo Videos: https://www.youtube.com/watch?v=SkbP5nQnRZc und 
% Satellitennavigation: https://www.youtube.com/watch?v=lPkETIy0P9E

\subsection*{WiMAX}
This technology is part of the IEEE 802.16 protocol family and like WLAN (IEEE 802.15) a radio technology. Unlike WLAN, WiMAX (Worldwide Interoperability for Microwave Access) can operate "at higher speeds, over greater distances and for a greater number of users." \cite{wimax} In particular, this technology works with partially interconnected base stations which are connected to the internet via high bandwidth, wired connection and are accessible by receivers in an area of around 8.000 square km.


\subsection*{Cellular Networks/GSM}

\section{Indoor Positioning}

This subsection presents various indoor positioning technologies and methods.
Whereas outdoor positioning typically relies on trilateration, indoor environments offer lots of alternatives.


\subsection*{WLAN}

\subsection*{Bluetooth}

\subsection*{RFID}

\subsection*{Ultrasound}

\subsection*{Frequency Modulation}



\section{Datamodels for Indoor Localization}

\subsection*{Set-based}
	
\subsection*{Hierarchical} 
	
\subsection*{Graph-based} 
	
\subsection*{Combined}

\section{Indoor-/Outdoor Transition Solutions}
\subsection*{Use GPS or WLAN until signal loss}
	
\subsection*{prefer WLAN over GPS}	
\subsection*{prefer GPS until signal loss (of 5s)}
	
\subsection*{vision/ user interaction-based approaches}
	
\subsection*{use SNR and GPS accuracy, dependent on going indoors or outdoors}

\section{Assessment and Conclusion}
\subsection*{Outdoor positioning}
		%Saltelliten, da z.b. sendemasten zu kleine gebiete abdecken können und außerdem in tälern (schattenseite) nicht positionieren
	A network of satellites is the only way to cover global range, as for example networked base stations would require partially unfeasible installation of transceivers anywhere on Earth (e.g. in oceans, mountains, dells and gorges etc.) to provide equal coverage.
	
	
\subsection*{Indoor positioning}
	
\subsection*{Datamodel}
	
\subsection*{Transition solution}
 combination of interaction and measurement


% END BACKGROUND % 