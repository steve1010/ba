\chapter{Related Work}

%TODO: overview to this section

% Verwandte Arbeiten und ihre Drawbacks.
This section presents and assess related works covering indoor/outdoor navigation and datamodels. Their respective drawbacks are determined and their approaches delimited in relation to this work.

% Indoor and Outdoor Positioning
The research area of indoor and outdoor positioning technologies and methods also forms part of this work, but this area is already explored sufficiently in recent years by various other researchers (\cite{seamlessIOsolutions}, \cite{surveyIPS}, \cite{recentAdvances}), such that no prototyping and testing is required anymore and one can use their well-confirmed evaluation results.

% darauf aufbauende Ansätze (meine Quellen für technologies and methods) 	
% Indoor positioning
% 1) Survey IPS
Among others surveying different technologies and methods for indoor positioning systems, \cite{surveyIPS} covers an evaluation of systems wrt. the criteria security and privacy, cost, performance, robustness, complexity, user preferences, commercial availability, and limitations. However, the authors' focus is on indoor localization only, aspects like I/O transitions or hybridization are ignored. Also, they have only regard to existent systems and used technologies and methods are not included in any evaluation tables, such that there is no satisfying overview to the systems' most relevant components (on one page).

% 2) Recent Advances in IL Techniques and Systems
In contrast to the survey discussed before, the authors of \cite{recentAdvances} examine single technologies like GPS, Infrared, WLAN, Ultrasound, RFID, BT, ZigBee and FM regarding the criteria accuracy, applicable positioning methods, coverage, power consumption and cost. Methods are evaluated wrt. measurement type, indoor accuracy, coverage, the requirement for line of sight, multipath affection and cost. Nevertheless, topics like hybridization of methods or combination with GPS technology and seamless IO transitions are not discussed.

% TODO: transition paper
% indoor example 3)

% indoor example 4)
%TODO: localization models paper?
% TODO: Outdoor positioning related work?

% Transitions I/O % Wie gestalten andere Indoor outdoor Übergänge? +Bewertung
% 1) bad examples / representatives
Some authors treating I/O transition only focus on single technologies and methods like \cite{indoorFingerprinting} (fingerprinting) and \cite{drear} (dead-reckoning), or basically only present their system design like in \cite{machineLearningIO} (machine learning approach), such that their foundings can not (or only partially) be used for a sensible overview and assessment of technologies and methods.

I/O transitions are also topic in \cite{transitionalSpaces}, but they focus on finding and presenting examples for transitional spaces and their properties only, and a answer to the question how to integrate this knowledge into a positioning system is not provided.

% bad example: Jinlong, 2013: Research on Seamless IO
A comparison of existent systems can be found in \cite{seamlessIOresearch}, but there only four existing systems are presented, which does not even yield a sensible overview - other authors assess more than ten solutions (\cite{seamlessIOsolutions}). Also, existing technologies, methods and transition strategies are poorly described - and a series of them even completely omitted.

% 2) good example 1) Streamspin (Hansen et al.2009)
Fortunately, various authors also provide sensible overviews to indoor/outdoor transitions, such as Hansen et al. (2009, Streamspin, \cite{streamspin}), who worked out and evaluated four different strategies for transition in I/O positioning with WLAN fingerprinting and GPS: 1) always prefer GPS, 2) always prefer WLAN, 3) prefer GPS until lost signal, then prefer WLAN until lost signal and 4) prefer GPS upon continuous readings (until signal loss of five seconds).
The authors conclude that strategy 4) is the most precise and reliable solution, which is also confirmed by \cite{seamlessIOresearch}.
However, they do not assess other technologies and methods, specially hybride low-cost approaches like with RFID or BT technology and approaches focusing on minimal configuration are left out. Furthermore, their system design is ineffizient, cause as soon as users pass a building (and do not enter it) its indoor radio map is downloaded, which could possibly lead to massive overhead.
This work's Section 4.2.1 gives deeper insight to drawbacks of the WLAN fingerprinting solution.

% 2) example: Maghdid et al. Seamless IO solitions for smartphones
\cite{seamlessIOsolutions} discusses available indoor positioning systems, outdoor positioning systems and also hybride approaches being able to seamlessly locate smartphones inside and outside buildings. Different technologies, methods and their combinations are also part of discussion.
The authors claim aspects like poor performance or accuracy, the absence of a plattform for integrating multiple positioning solutions, and demand infrastructureless, cooperative solutions and hybridization of methods and technologies in combination with sensor fusion and Kalman filters.

However, they do not implement their own positioning system and focus on evaluating other solutions. Their aim is to identify weaknesses and opportunities and upcoming challenges wrt. seamless and precise IO positioning.

% 3) seamlessGPShandoverStrategy

Another related work is \cite{seamlessGPShandoverStrategy}, where the handover from GPS to an indoor localization technology is faced. The authors experimentally found that GPS signal loss might be an insufficient indicator to mark outdoor/indoor handover points. The \ac{snr} is rather a better choice, as "SNRs of the specific satellites rapidly decrease around building entrances". This refers to the fact that satellites at elevation of 30-90 degrees show significantly higher \ac{snr} drop when entering a building than those at 0-30 degrees (sometimes even no significant changes are apparent).
SNR does not provide any significance in case of indoor/outdoor handovers, as the values it supplies outdoors are almost equal to values measurable indoors near windows. In this case, GPS accuracy attribute turned out to be decisive. 
More details on their foundings and other GPS fundamentals are rehashed Secton 4.
Unfortunately, they do not evaluate the aid of indoor localization technologies and methods and other possibilities to determine the best handover time and only focus on GPS technology. In addition, their current system design is worthy of improvement: it requires a user to visit a respective building at least once before being fully operable. Cooperative strategies would represent a promising alternative to that issue.
%and use GPS even indoors (but with reduced sampling interval).

% heutiger Stand: AI and fuzzy logic
Papers published in the past two years mostly use advanced methods like fuzzy logic, artificial intelligence (\cite{ai1}), sensor fusion (\cite{sensorFusion1}) and extended Kalman or particle filters to overcome noisy measurements and improve accuracy. One of the drawbacks of such techniques is their complexity and high calculation effort, such that the required computing power can either not be provided on smartphones or is only feasible under massive energy consumption.\\




%  conclusion
As shown in this section, some articles in this area only consider single technologies and methods but not hybridization (of methods or IO technologies) and IO transitions.
Most papers also do not show an interest in presenting the invented datamodel for localization systems.

This work catches up with this issue and examines indoor-/outdoor transition and positioning strategies as well as relevant datamodels with the aim to provide proposals for various system requirements.\\


% END RELATED WORK SECTION